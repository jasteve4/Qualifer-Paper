\begin{filecontents*}{\jobname.bib}
@techreport{Asanović:EECS-2016-17,
    Author = {Asanović, Krste and Avizienis, Rimas and Bachrach, Jonathan and Beamer, Scott and Biancolin, David and Celio, Christopher and Cook, Henry and Dabbelt, Daniel and Hauser, John and Izraelevitz, Adam and Karandikar, Sagar and Keller, Ben and Kim, Donggyu and Koenig, John and Lee, Yunsup and Love, Eric and Maas, Martin and Magyar, Albert and Mao, Howard and Moreto, Miquel and Ou, Albert and Patterson, David A. and Richards, Brian and Schmidt, Colin and Twigg, Stephen and Vo, Huy and Waterman, Andrew},
    Title = {The Rocket Chip Generator},
    Institution = {EECS Department, University of California, Berkeley},
    Year = {2016},
    Month = {Apr},
    URL = {http://www2.eecs.berkeley.edu/Pubs/TechRpts/2016/EECS-2016-17.html},
    Number = {UCB/EECS-2016-17},
    Abstract = {Rocket Chip is an open-source Sysem-on-Chip design generator that emits synthesizable RTL. It leverages the Chisel hardware construction language to compose a library of sophisticated generators for cores, caches, and interconnects into an integrated SoC. Rocket Chip generates general-purpose processor cores that use the open RISC-V ISA, and provides both an in-order core generator (Rocket) and an out-of-order core generator (BOOM). For SoC designers interested in utilizing heterogeneous specialization for added efficiency gains, Rocket Chip supports the integration of custom accelerators in the form of instruction set extensions, coprocessors, or fully independent novel cores. Rocket Chip has been taped out (manufactured) eleven times, and yielded functional silicon prototypes capable of booting Linux.}
}
\end{filecontents*}
\documentclass[10pt,conference]{IEEEtran}
\IEEEoverridecommandlockouts
% The preceding line is only needed to identify funding in the first footnote. If that is unneeded, please comment it out.
\usepackage{cite}
\usepackage{amsmath,amssymb,amsfonts}
\usepackage{algorithmic}
\usepackage{graphicx}
\usepackage{textcomp}
\usepackage{xcolor}
\usepackage[english]{babel}
\usepackage{blindtext}
\def\BibTeX{{\rm B\kern-.05em{\sc i\kern-.025em b}\kern-.08em
    T\kern-.1667em\lower.7ex\hbox{E}\kern-.125emX}}
\begin{document}

\title{Using Rocket Chip in Chips SOC
}

\author{\IEEEauthorblockN{Joshua Stevens}
\IEEEauthorblockA{\textit{Electrical and Computer Engineering Department} \\
\textit{North Carolina State University}\\
Raleigh, US \\
jasteve4@ncsu.edu}
}

\maketitle

\begin{abstract}
\blindtext
\end{abstract}

\begin{IEEEkeywords}
component, formatting, style, styling, insert
\end{IEEEkeywords}
\section{Introduction}
\blindtext 

\blindtext

\section{Chips Architecture}
\blindtext

\subsection{Chips SOC}
\blindtext
\subsubsection{Sub-System}
\blindtext
\subsubsection{2.5D Structrue}
\blindtext
\subsection{Rocket Chip RISC-V Processor}
\blindtext
\subsection{Chips Accelerators}
\subsubsection{LSTM}
\blindtext
\subsubsection{SCNN}
\blindtext

\section{UCB-BAR: Rocket Chip Generator}
\blindtext
\subsection{Chisel DSL}
\blindtext
\blindtext
\subsection{Rocket Chip Configuration}
\blindtext
\subsection{Chips Configuration}
\blindtext
\subsection{Base Configuation}
\blindtext
\subsection{XBus Controller}
\blindtext
\section{Testing Environment}
\blindtext
\subsection{RTL Simulator}
\blindtext
\subsection{Linux Interface}
\blindtext

\section{High Level Programming support}
\blindtext

\section{Related Works}
\blindtext
\subsection{NVDLA}
\blindtext

\section{Conclusion}
\blindtext

\blindtext

%\begin{thebibliography}{00}

%\end{thebibliography}
\bibliographystyle{IEEEtran}
\bibliography{IEEEabrv,ref}
%\vspace{12pt}
%\color{red}
%IEEE conference templates contain guidance text for composing and formatting conference papers. Please ensure that all template text is removed from your conference paper prior to submission to the conference. Failure to remove the template text from your paper may result in your paper not being published.

\end{document}
