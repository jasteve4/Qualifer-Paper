\documentclass[../main.tex]{subfiles}

\begin{document}
\begin{abstract}
Chip development in edge applications can take a significant amount of time to design and validate. In each new chip development cycle, there is a partial amount of intellectual property (IP) reuse and the addition of some new IP. New IP usually takes the form of an application-specific integrated circuit (ASIC). When new IP is introduced into the system it can lead to bugs and extend development cycle. The CHIPS (common heterogeneous integration and IP reuse strategies) project leverages the idea of IP reuse at the post die phase of the development cycle. Dies at this stage are considered to be chiplets, and each chiplet connects to an interposer and then connects to another chiplet. This method allows new IP to be developed separately from the old IP and reduces the development cycle needed to bring a new chip to market.In phase one, the Mie Fujitsu 55nm LP process was used to fabricate the die, and this phase is a debug run, so the interposer package becomes omitted. Connects that would have been in the interposer are now routed in the standard metal stack.
\end{abstract}
\end{document}