\documentclass[../main.tex]{subfiles}

\begin{document}
\section{Conclusion}
%The CHIPS project proposed the idea of integrating chiplets after fabrication. The methodology allowed the following chiplets to be connected post-fabrication: Rocket, LSTM, SCNN. Due to constraints in the area and time, only one chiplet has AIB wrapped around its IO. The Rocket chiplet is the idea choice because of its size and design maturity. The UCB taped out the rocket chip design sever time over the past few years\cite{Asanović:EECS-2016-17}. So, wrapping the Rocket chip is the least risky chiplet to wrap with AIB. Due to no interposer, AIB needs to connect through standard metal layers. All other chiplets use standard CMOS logic for its IO.

%his paper forces on the idea of how to use the rocket chip in an SoC designed to be modular. It covers how the Rocket chip generator creates the Rocket chiplet and what modifications were made to integrate the Rocket chiplet into the CHIPS SoC. This paper details how to develop and run a high-level code for the CHIPS architecture. 

The reuse of intellectual property without re-fabrication has the potential to save significant amounts of design and validation time. The 2.5D structure allows chiplet integration at the packaging phase, and this strategy enables chiplet design and validation to be separate from the system design. One downside to this strategy is that all chiplets have to share a standard interface or support a multitude of interfaces, which can lead to more complicated designs. The standard interface is a problem because it can lead to a reduction in functionality in the ASIC. If all memory chiplets are 32-bit data, and the ASIC chiplet needs 128-bit data, then this chiplet will starve on memory accesses.  Another downside is increases in network latency over the AIB interface. This strategy causes a trade-off between network latency and design and validation time. 

The preceding section has covered how to interface and control non-standalone heterogeneous chiplets in the CHIPS architecture. These sections include: generating Rocket chiplet, how chiplets get managed, and how they communicate with each other, how the architecture gets simulated, how programs interface with non-standalone chiplets, and what process node was used to implement this architecture. 

In phase two, the project will implement an interposer. The interposer will force all chiplets to use the AIB interface on its IO. The support structure will be removed and replaced with an AXI router on each chiplet, or this structure will need to become a chiplet of its own. The latter will be the most likely choice because it will allow chiplets to connect on an ad-hoc basis.
\end{document}