
\documentclass[../main.tex]{subfiles}
\begin{document}
\section{Introduction}
Considerable time and effort go into designing and validating a chip, and with every new chip, there is the addition of some new functionality. New functionality often comes in the form of application-specific integrated circuit (ASIC) designs, and any system that communicates with these ASICs will form a heterogeneous system on-die, where most of the designs are left unchanged. Most of the effort in design goes to the ASIC and integrating it with the rest of the system. Because the system is so complex, a failure in any part of the system will result in a chip-wide failure. The Defense Advanced Research Projects Agency project CHIPS (common heterogeneous integration and IP reuse strategies) aims to mitigate the cost of designing and validating chip on-die by changing heterogeneous system on-die to heterogeneous system on-package. The system on-package is known as 2.5D structure, and the technology that allows the dies to communicate through the package is an advanced interface bus (AIB). 

Phase one of the CHIPS project proposes the following chiplets: long short term memory (LSTM), sparse convolution neural network (SCNN), and Rocket processor. The LSTM chiplet is a result of the Ph.D. work done by Dr. Sumon Dey, the SCNN chiplet is a result of the Ph.D. work done by Dr. Weifu Li, and the Rocket chiplet is a result of the University of California, Berkeley (UCB-BAR) Rocket chip Generator (RCG). In this phase, there is no interposer, so the AIB structure connects through standard metal layers, and for risk mitigation, the Rocket chiplet is the only one to have AIB. This paper focuses on two aspects of the CHIPS project: the role of the general-purpose processor in the CHIPS architecture and phase one implementation of this architecture. The sections are as follows: \ref{sec:RCG} \nameref{sec:RCG}, \ref{sec:CHIPS-Arch} \nameref{sec:CHIPS-Arch}, \ref{sec:sim-env} \nameref{sec:sim-env}, \ref{sec:software-support} \nameref{sec:software-support}, and \ref{sec:Chip-summary} \nameref{sec:Chip-summary}.
\end{document}