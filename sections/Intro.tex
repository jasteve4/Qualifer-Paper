
\documentclass[../main.tex]{subfiles}
\begin{document}
\section{Introduction}
Chip designs can take a long time to develop and even longer time to validate. The CHIPs project proposes the concept of chiplets. A chiplet is a self-contained design that uses a standard interface to connect to a more extensive system. This standard interface must be able to communicate through an interposer and must be validation clean and before the packaging stage. In this stage, the chiplets can be connected to an interposer, and finally connected to a standard package and ready for use. Because the chiplets have passed the validation phase, the only validation need is the connection to the interposer. This concept can decrease the time to market by reducing the time to develop and validate. 

The CHIPs project proposes the following chiplets:  long short term memory (LSTM), sparse convolution neural network (SCNN). In addition to the neural network chiplets, there is a host chiplet, the Rocket chip. The Rocket chiplet provides externe control for the neural network chiplets. This paper will cover the fundamental aspects of the Rocket chiplet. There five-section in this paper, not including this section. The sections are as follows. Two, Rocket chip generator, details the default system on chip (SOC) created by the generator. Three, CHIPs architecture, details physical structure, SOC, and chiplets used in this architecture. Four, simulation environment, details the development and testing environment used to validate the CHIPs architecture. Five, Results, covers the results of the fabrication of the CHIPs system.
\end{document}