
\documentclass[../main.tex]{subfiles}
\begin{document}
\section{Introduction}
Chip designs can take a long time to develop and even longer time to validate. The CHIPS project proposes the concept of integrating heterogeneous chiplets that can be connected post-design. A chiplet is a self-contained design that uses a standard interface to connect to a more extensive system. This standard interface must be able to communicate through an interposer package and must be validated up to the packaging stage. In the packaging stage, chiplets get attached to an interposer package. At this stage, when the chiplets connect through an interposer, the overall design becomes the sum of its parts.  This concept can decrease the time to market by reducing the time to develop and validate individual designs. 

The CHIPs project proposes the following chiplets:  long short term memory (LSTM), sparse convolution neural network (SCNN). In addition to the neural network chiplets, there is a host chiplet, the Rocket chiplet. The Rocket chiplet provides external control over the neural network chiplets. This paper will cover the fundamental aspects of the Rocket chiplet, and the paper follows the following layout. Section two, Rocket chip generator, details the genetic system on chip (SoC) created by the Rocket chip generator. Section three, CHIPS architecture, details physical structure, SoC, and chiplets used in CHIPS architecture. Section four, simulation environment, details the development and testing environment used to validate the CHIPs architecture post-silicon. Section five, results, covers the fabrication results of the CHIPS design.
\end{document}