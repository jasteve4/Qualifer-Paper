
\documentclass[../main.tex]{subfiles}
\begin{document}
\section{Results}
This project uses the 55nm FUJISC process and the chip fabrication followed a hierarchy flow. This flow treats each chiplet as a separate macro. Macros are stitched together at the top level of the chip design. A summary of the size, area, and frequency of each chiplet is in table \ref{tab:results}. Each chiplet had a core clock frequency of sub 250MHz, but when the chiplet got tied together, the core clock frequency dropped to 50MHz. The top-level chip has a size of 10mm by 10mm and an area of 100mm\textsuperscript{2}. Figure 2 shows the final layout of this project.

\begin{center}
\begin{tabular}{||c c c c||}
 \hline
 Chiplete & Size & Area & Frequency \\ [0.5ex] 
 \hline\hline
 Rocket Core & 6 & 87837 & 787 \\ 
 \hline
 SCNN & 7 & 78 & 5415 \\
 \hline
 LSTM & 545 & 778 & 7507 \\
 \hline
 Main Memory & 545 & 18744 & 7560 \\
 \hline
 CIPI Bridge & 88 & 788 & 6344 \\ [1ex] 
 \hline
\end{tabular}
\label{tab:results}
\end{center}

\end{document}