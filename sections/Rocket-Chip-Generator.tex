\documentclass[../main.tex]{subfiles}

\begin{document}

\begin{figure}
    \centering
    \includegraphics[scale=.5]{pngs/RocketChipGeneratorLayout.png}
    \caption{Rocket Chkp Pipeline\cite{Asanović:EECS-2016-17}}
    \label{fig:RocketCipGen}
\end{figure}

The Rocket Core was generated by using the Rocket Chip Generator. The Rocket Chip Generator was developed by UC. Berkeley and is maintained by the CHIPS Alliance. Figure \ref{fig:RocketCipGen} shoe a high level overview of a default Rocket Chip. 

Starting form top to bottom: The Generator can be used to create N Number of Tiles. Tiles are the most basic building block in the system. It contains a processor, L1 instruction cache, L1 data cache and ROCC Accelerator. The ROCC Accelerator can be any custom function that is required for a specific application. One of noteworthy accelerator is the Hwacha Vector-Fetch Architecture\cite{HwachaPaper}. Each Tile is connected together by the L1toL2 Network. Next, is the L2 Cache banks. The L2 Cache banks are connected to L2toIo Network. This network is entry point for the Rocket Core in to a global system.

\subsection{RISC-V Processor}
\begin{figure}
    \centering
    \includegraphics[scale=.4]{pngs/RocketPipeline.png}
    \caption{Rocket Chkp Pipeline\cite{Asanović:EECS-2016-17}}
    \label{fig:RocketCipFlow}
\end{figure}
The default RSIC-V Processor is called the Rocket Core. The Rocket Core is a single fetch, single issue, in-order scalar processor. The Rocket Core was designed to the RISC-V ISA standard. Figure \ref{fig:RocketCipFlow} show the basic flow of the Rocker Core. The basic version of the Rocket Core use the integer (I) extension. It can be configured to support additions extension: multiply and divide (M), atomics (A), single-precision (F) and double-precision (D) floating point\cite{Asanović:EECS-2016-17}. This set of extension are know as (IMAFD), or (G) extension\cite{Asanović:EECS-2016-17}. To support accelerators, the Rocket Core use the (E) extension. These set of instruction control the ROCC interface. The ROCC interface interfaces the Rocket Core and the ROCC Accelerator.
\subsection{Chisel}
Chisel3 is an domain specific language (DSL) written Scala. Chisel is used to embed RTL in a high level language. It combines function program with hardware description. The Rocket Chip Generator is written in this DSL. One of the advantage to this DSL, is that port list can grow and shrink based off the configuration, and hardware generation can be controlled with complex function. This can be done with the use of defines in verilog, but it would not be salable. There would need to be a different define for each reparation of code. Interface can be used, but they can not be extended or reduced. Interface can be parameterized, but this function not wildly supported across all synthesis tools.  
\subsection{Configuring the Rocket Chip Generator}
The Rocket Chip Generator uses a set of configures to generator the Rocker Chip in verilog. Each module in the Rocket Chip Generator take a set of configuration. Depending on which set of configuration are set some moudle will be left out. For example, to have a accelerator added in a tile, extend the base configure and add the configuration for the accelerator. The main configuration file is located in the following directory: /src/main/scale/system/Config.scala. Figure ?? show an example of such a configuration.
\end{document}