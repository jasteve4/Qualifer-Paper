\documentclass[../main.tex]{subfiles}
\begin{document}

\section{Simulation Environment}
The simulation environment is used to simulate the whole chip and the off chip commutation. The simulation environment consists of two basic parts: RTL simulator, and the Linux environment.

\subsection{RTL Simulator}
The RTL simulator handles the validation of the chip and its peripherals. The chip's sub-module, chipletes, can be simulated in one of the following stages: pre-synthesis, post-synthesis and post-place-and-routed. So, each chiplete can be tested form any of the stages with out any of the other chipletes cause a problem. The chip's peripherals are seven Jtag bus and one CIPI bus. The CIPI bus is connected to a fake CIPI Memory controller. The fake CIPI memory controller emulates an access to DRAM. Depending on which chiplete being tested, a Jtag controller will be connected to it, and all the other Jtag bus are tied of. The Jtag controller was created in to parts: the RLT interface and the c++ DPI interface. The RTL interface created a bridge between the RTL simulator and the c++ DPI code. The c++ code crated a tcp port for the linux simulator to connect to.  
\subsection{Linux Environment}
The Linux Environment can be divided into three parts: the RTL simulator, on chip debugger (OCD) sever, and end user program. All parts are connected by tcp ports. As descried in the section before, the RTL simulator creates a Jtag over tcp port. The OCD server attaches to port created by the RTL simulator and creates two new ports: GNU Project Debugger (GDB) tcp port and applications tcp port. The GDB port can be used interactive to interact with the Rocket core. The applications port allows scripts to interact with any of the chipletes. OCD provides a python library to interface with the OCD server.

\end{document}