\documentclass[../main.tex]{subfiles}
\begin{document}
\section{Simulation Environment}
\label{sec:sim-env}
The simulation environment simulates the interface between the chip and its peripherals. It uses three different parts: HDL simulator, end-user program, and open on-chip debugger (OCD).  All three run independently in the Linux environment.

In the Linux environment, all parts connect through UDP port. The following sequence initiates the simulation environment. First, HDL simulation creates a UDP port, Jtag UDP. Next, OCD is stated and connects to the first UDP port created by the HDL simulator. When OCD stats, it creates two new UDP ports: end-user port and  Gnu debug bus (GDB) port. The GDB port can be used to interactive run and load programs onto the rocket core. The end-user port is used to give a sequence of commands to control any of the chiplets.



\end{document}