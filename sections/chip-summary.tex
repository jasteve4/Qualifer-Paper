
\documentclass[../main.tex]{subfiles}
\begin{document}
\section{Chip Summary}
%The CHIPS project used the Fujitsu 55nm process to fabricated the final chip. Each chiplet used the Synopsys design compiler to synthesize its design. Then, the chiplet used the Cadences Innovus tool to place and route the design. VCS was used to verify the functionality of the post place and routed chiplets. Cadences Tempest timing tool was used to verify the hold and setup timing in the post place and routed design. Cadence Voltus tool was used to get dynamic and static power of each chiplet. Ones each chiplet was placed and routed, the top-level design, including each chiplet and IO driver, was pushed through the same process. Figure 1 shows the final CHIPS layout. The final chip core clock frequency is around 50MHz and has a size of 10mm by 10mm and an area of 100mm\textsuperscript{2}. Table 1 shows the size, area, and power for each of the chiplets.

The CHIPS architecture uses the Mie Fujitsu 55nm LP process. Figure ?? shows the chip die photo with chiplets and support structures are highlighted. Each chiplet can operate at 200Mhz core clock rate, and with the support structure, the core clock rate reduces to 50Mhz. All chiplets and support structures are projected to work at the typical typical corner, at 0.8-V digital supply voltage (VDD) and 40 degrees Celsius. 

%\begin{figure}
%    \centering
%   \includegraphics[scale=.06]{pngs/CHIPS-Layout.png}
%    \caption{CHIPS Layout}
%    \label{fig:CHIPS-layout}
%\end{figure}

\end{document}